\begin{frame}

\frametitle{Functional Completeness}

\begin{definition}

Given a set $\mathbb{B} = \set{0,1}$, a set of functions $F=\set{f_i :
\mathbb{B}^{n_i} \rightarrow \mathbb{B}}$ is \emph{functionally complete} if
all functions $f : \mathbb{B}^n \rightarrow \mathbb{B}$, for all $n \geq 1$,
can be generated by the functions in $F$.

\end{definition}

Examples of fnctionally complete sets:

\begin{itemize}

\item $\set{\text{\textbf{NAND}}}$, $\set{\text{\textbf{NOR}}}$,

\item $\set{\text{\textbf{AND}},\mathrel{}\text{\textbf{NOT}}}$,
$\set{\text{\textbf{OR}},\mathrel{}\text{\textbf{NOT}}}$.

\end{itemize}

\begin{center}

\textbf{Consequence}: Any boolean function can be built by combining functions
from a functionally complete set.

\end{center}

\end{frame}
